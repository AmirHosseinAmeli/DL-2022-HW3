\begin{enumerate}[label=(\alph*)]
	
	\item
	
	پدیده Collapse Mode در شبکه های GAN ، زمانی اتفاق می افتد که قسمت generator تنها قادر به تولید یک نوع خروجی یا طیف بسیار محدودی از خروجی ها می باشد. این امر می تواند به دلیل آموزش نامناسب شبکه اتفاق بیفتد یا در حالتی که شبکه generator طیف کوچکی از داده ها را بیابد که به سادگی توان فریب discriminator را دارند. 
	برخی از راه حل های پیشنهادی برای حل collapse mode به شرح زیر می باشد:
	
	\begin{itemize}
		
		\item
		استفاده از معماری W-GAN 
		\item
		استفاده از روش Unrolling : به روز رسانی وزن های generator پس از k مرحله از به روز رسانی وزن های discriminator. اینکار باعث می شود شبکه generator تا چند مرحله از آینده را رصد کرده و سپس تشویق به تولید خروجی های متنوع تری شود.
		\item
	استفاده از روش Packing : ارتقا دادن discriminator به گونه ای که تصمیم خود را بر اساس چندین نمونه از یک کلاس برای تشخیص جعلی یا حقیقی بودن اتخاذ کند.
		
		
	\end{itemize}


	\item
	
	پاسخ این سوال به صورت مشروح در
	\href{https://arxiv.org/pdf/1701.07875.pdf}{این لینک}
	داده شده است. توجه شود پاسخی خلاصه بر مبنای این مقاله یا توضیحات داخل جزوه کاملا پذیرفته بوده و نمره کامل خواهد گرفت.
	
\end{enumerate}