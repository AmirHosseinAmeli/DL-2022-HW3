% !TeX program = xelatex

\documentclass[10pt,a4paper]{article}
\usepackage[inline]{enumitem}
\usepackage{pgfplots}
\usetikzlibrary{calc}
\newcommand{\drawaline}[4]{
	\draw [extended line=1cm,stealth-stealth] (#1,#2)--(#3,#4);
}
\usepackage{float}
\usepackage{tikz}
\usetikzlibrary {positioning}
%\usepackage {xcolor}
\definecolor {processblue}{cmyk}{0.96,0,0,0}

\usetikzlibrary{automata,arrows.meta}

\usepackage{commons/course}
\usepackage{hyperref}
\usepackage{multirow}
\usepackage{graphicx}
\usepackage{neuralnetwork}

\hypersetup{
	colorlinks=true,
	linkcolor=cyan,
	filecolor=blue,      
	urlcolor=magenta,
}


%\hidesolutions

\شروع{نوشتار}

\سربرگ{تمرین سری چهارم}{موضوع!}{موعد تحویل: دی}{مدرس: دکتر بیگی}

\begin{itemize}
	\small
	\setlength\itemsep{0.05em}
	\item
	مهلت ارسال پاسخ تا ساعت 23:59 روز مشخص‌شده است.
	%	\item
	%	در طول ترم امکان ارسال با تاخیر  پاسخ همه‌ی تمارین (به استثنای هفته‌ی امتحان میانترم) تا سقف پنج روز و در مجموع ۱۵ روز، وجود دارد. پس از گذشت این مدت، پاسخ‌های ارسال‌شده پذیرفته نخواهند‌بود. 
	\item
	هم‌کاری و هم‌فکری شما در انجام تمرین مانعی ندارد اما پاسخ ارسالی هر کس حتما باید توسط خود او نوشته شده‌ باشد. 
	\item
	در صورت هم‌فکری و یا استفاده از هر منابع خارج درسی، نام هم‌فکران و آدرس منابع مورد استفاده ‌برای حل سوال مورد نظر را ذکر‌ کنید.
	\item
	لطفا تصویری واضح از پاسخ سوالات نظری بارگذاری کنید. در غیر این صورت پاسخ شما تصحیح نخواهد شد.
\end{itemize}


\section*{سوالات نظری (5 + 60 نمره)}

\مسئله{(10 نمره)‌}
مکانیزم توجه برای از بین بردن گلوگاه اطلاعات بین رمزگذار و رمزگشا معرفی شده است. به این صورت که به جای آخرین بردار نهان رمزگذار، رمزگشا به تمام بردارهای نهان رمزگذار دسترسی دارد. این مکانیزم به صورت زیر فرموله می‌شود و در هر گام شبکه‌ی تکرارشونده‌ی رمزگشا مورد استفاده قرار می‌گیرد:
\begin{equation}
	{a_t}\left( s \right) = \frac{{\exp score\left( {h_d^{\left( t \right)},h_e^{\left( s \right)}} \right)}}{{\sum\limits_{s'} {\exp score\left( {h_d^{\left( t \right)},h_e^{\left( {s'} \right)}} \right)} }}
\end{equation}
\begin{equation}
	{c_t} = \sum\limits_{s'} {{a_t}\left( {s'} \right)h_e^{\left( {s'} \right)}} 
\end{equation}
\begin{equation}
	\hat h = \tanh {W_c}\left[ {{c_t};h_d^{\left( t \right)}} \right]
\end{equation}
\begin{equation}
	{y_t} = softmax \left( {{W_s}\hat h} \right)
\end{equation}

که در آن
${h_d^{\left( i \right)}}$
بردار نهان رمزگشا،
${h_e^{\left( i \right)}}$
بردار نهان رمزگذار و
$y_t$
خروجی گام
$t$
ام رمزگشا می‌باشد. تابع
${score\left( {h_d^{\left( t \right)},h_e^{\left( s \right)}} \right)}$
را می‌توان به سه روش زیر تعریف کرد:
 
$$
score\left( {h_d^{\left( t \right)},h_e^{\left( s \right)}} \right) = \left\{ {\begin{array}{*{20}{c}}
		{h_d^{{{\left( t \right)}^T}}h_e^{\left( s \right)}}&{dot}\\
		{h_d^{{{\left( t \right)}^T}}{W_a}h_e^{\left( s \right)}}&{general}\\
		{v_a^T\tanh {W_a}\left[ {h_d^{\left( t \right)};h_e^{\left( s \right)}} \right]}&{\tanh layer}
\end{array}} \right.
$$

\begin{enumerate}[label=(\alph*)]
	\item
	این سه تابع را از نظر توان مدل کردن، هزینه‌ی محاسباتی و عبور گرادیان در مرحله بازانتشار خطا مقایسه کنید. شما کدام یک را برای یک شبکه
	\lr{Seq2Seq}
	انتخاب می‌کنید؟
	\item
	در ادبیات یادگیری عمیق، دو کار پژوهشی دو مکانیزم توجه ارائه داده‌اند که جز رایج‌ترین کار‌های این حوزه می‌باشد: 1-
	\href{https://arxiv.org/abs/1409.0473}{مکانیزم1}
	2-
	\href{https://arxiv.org/abs/1508.04025}{مکانیزم2}
	.
	این دو ساختار را با هم مقایسه کنید و تفاوت‌های آن را ذکر کنید. کدام یک توانایی مدل کردن بیشتری دارد؟
	\item
	یکی از مشکلات رایج مکانیزم توجه، مخصوصا هنگامی که متن ورودی در طرف رمزگذار طولانی باشد، عدم توانایی این مکانیزم در پرداختن به تکه‌های مختلف متن ورودی است. به طور مثال ممکن است در تمامی گام‌های رمزگشا، مکانیزم توجه فقط به یک یا دو کلمه ی خاص امتیاز بسیار بالایی بدهد و فقط آن‌ها را در نظر بگیرد. در این صورت مدل قادر نخواهد بود که از تمامی متن ورودی استفاده کند. برای حل این مشکل چه راهکاری پیشنهاد میدهید؟ توضیح دهید.
\end{enumerate}

%\حل{
%	\begin{enumerate}[label=(\alph*)]
	\item
	
	با نوشتن رابطه امیدریاضی و استفاده از
	\href{https://en.wikipedia.org/wiki/Leibniz_integral_rule}{قانون انتگرال لایبنیتز}
	 داریم:
	$$
	\begin{aligned}
		{\nabla _\theta }{\mathbb{E}_{z\~{q_\theta }\left( z \right)}}\left[ {f\left( z \right)} \right] &= {\nabla _\theta }\int_z {f\left( z \right){q_\theta }\left( z \right)dz} \\
		&= \int_z {f\left( z \right){\nabla _\theta }{q_\theta }\left( z \right)dz} 
	\end{aligned}
	$$
	
	‌سپس با توجه به تساوی
	${\nabla _\theta }{q_\theta }\left( z \right) = {q_\theta }\left( z \right){\nabla _\theta }\ln {q_\theta }\left( z \right)$
	داریم:
	$$
	{\nabla _\theta }{\mathbb{E}_{z\~{q_\theta }\left( z \right)}}\left[ {f\left( z \right)} \right] = \int_z {f\left( z \right){q_\theta }\left( z \right){\nabla _\theta }\ln {q_\theta }\left( z \right)dz} 
	$$
	
	سپس با تخمین مونته کارلو امیدریاضی بالا به مطلوب سوال دست پیدا می‌کنیم.\\
	مشکلی که تخمین بالا دارد این است که با اینکه تخمینگر نااریبی می‌باشد ولی طبق نتایج تجربی واریانس بالایی دارد و تابع $f$ در گرادیان تاثیر مستقیمی ندارد. برای کاهش واریانس ساده‌ترین روش افزایش مقدار N می‌باشد ولی باعث افزایش هزینه و محاسبات می‌شود.
	
	\item
	
 	باتوجه به اینکه هدف ما بیشینه‌سازی امیدریاضی تابع $f$ می‌باشد، در تابع هزینه بالا گرادیان‌های بدست آمده براساس نمونه‌‌های بدست آمده از توزیع کدگذار می‌باشد و تابع f دخالت مستقیمی در بهینه‌سازی ندارد. به عبارت دیگر مسیر بهینه‌سازی و گرادیان ‌های بدست آمده تنها وابستگی به نمونه‌های تولید شده دارد و ممکن است بعضی از نمونه‌هایی تاثیر مهمی دارند در فرایند نمونه‌گیری ظاهر نشوند. ولی در Reparameterization شرایط متفاوت می‌شود. شما فرض کنید توزیع کدگذار گاوسی می‌باشد. تابع هزینه بالا به صورت زیر تغییر پیدا می‌کند.
 	
 	$$
 	{\nabla _\theta }{\mathbb{E}_{z \sim \;{q_\theta }\left( z \right)}}\left[ {f\left( z \right)} \right] = {\nabla _\theta }{\mathbb{E}_{v \sim \;N(0,1)}}\left[ {f\left( {{\sigma _\theta }\left( x \right).v + {\mu _\theta }\left( x \right)} \right)} \right]
 	$$
	
	همانطور که در رابطه بالا مشخص است، در بدست آوردن گرادیان نسبت به پارامتر تتا، تابع $f$ نیز تاثیر مستقیمی می‌گذارد و باعث می‌شود گرادیان های بدست آمده بهتر از حالت اول باشند.
	
	\item
	
	در حالتی که فضای ما پیوسته‌ باشد و توزیع داده شده در حالت کلی یک توزیع چندمتغیره گاوسی با واریانس $I$ باشد،‌ با جایگذاری در رابطه بالا یک عبارتی به شکل خطای MSE ظاهر می‌شود.
\end{enumerate}
%}



\مسئله{(15 نمره)}
یکی از مشکلاتی که
\lr{transformer}
ها دارند این است که مرتبه هزینه محاسباتی و هزینه ذخیره‌سازی عملیات
\lr{self-attention}
دارای عبارت
$N^2$
می‌باشد. این مرتبه باعث می‌شود که آموزش این شبکه روی داده‌های طولانی مانند کتاب مشکل‌زا باشد. دلیل این امر عملگر
\lr{Softmax}
می‌باشد که برای محاسبه شباهت دو بردار استفاده می‌شود. در این تمرین قصد داریم به بررسی یک راهکار جایگزین برای این مورد بپردازیم. یکی از این راهکار‌ها استفاده از مکانیزم‌های توجهی کرنلی می‌باشد.

اگر ورودی را
$x \in {\mathbb{R}^{N \times F}}$
و ماتریس‌های مکانیزم توجه را
${W_Q} \in {\mathbb{R}^{F \times D}},{W_K} \in {\mathbb{R}^{F \times D}}$
و
${W_v} \in {\mathbb{R}^{F \times M}}$
در نظر بگیریم. می‌توان این عملیات را به صورت زیر نوشت:
$$
\begin{array}{l}
	Q = x{W_Q},K = x{W_K},V = x{W_V}\\
	V' = softmax \left( {\frac{{Q{K^T}}}{{\sqrt D }}} \right)V
\end{array}
 $$
 
 حال با تعریف
 $sim\left( {{Q_i},{K_j}} \right) = \exp \left( {\frac{{Q_i^T{K_j}}}{{\sqrt D }}} \right)$
 می‌توان این عبارت را به فرم زیر بازنویسی کرد:
 
\begin{equation} \label{softmax}
	{V_i}^\prime  = \frac{{\sum\limits_{j = 1}^N {sim\left( {{Q_i},{K_j}} \right){V_j}} }}{{\sum\limits_{j = 1}^N {sim\left( {{Q_i},{K_j}} \right)} }}
\end{equation}
 
\begin{enumerate}[label=(\alph*)]
	\item
	مرتبه زمانی و حافظه مورد نیاز برای محاسبه عملگر
	\lr{self-attention}
	بالا را براساس پارامتر‌های
	\lr{N,D,M}
	محاسبه کنید.
	
	\item
	یکی از توابعی که می‌توان جایگزین
	$sim\left( {{Q_i},{K_j}} \right)$
	کرد، کرنل توجه چند‌جمله‌ای می‌باشد. عبارت جایگزین را برای حالت درجه دو
	\lr{(Quadratic)}
	بنویسید.
	\item
	برای کرنل مرتبه بخش قبل، بردار ویژگی
	$\phi \left(  \cdot  \right)$
	را بنویسید
	\item
	حال باتوجه به رابطه
	${\rm K}\left( {q,k} \right) = \phi {\left( q \right)^T}\phi \left( k \right)$
	، رابطه
	\ref{softmax}
	را بازنویسی کنید و مرتبه زمانی رابطه و مرتبه حافظه مورد نیاز را برای آن محاسبه کنید. با مقایسه این مرتبه‌ها با مرتبه‌های رابطه قبلی، در چه شرایطی استفاده از این رابطه بهتر از رابطه قبلی می‌باشد؟
	
\end{enumerate}

%\حل{
%
%	\begin{enumerate}[label=(\alph*)]
	
	\item
	فرض کنید ورودی
	$x \in \mathbb{R}{^D}$
	به انکودر داده شده و z به دست آمده است. خروجی شبکه دیکودر را به صورت
	$a\left( z \right) = \sigma \left( {f\left( z \right)} \right) \in \mathbb{R}{^D}$
	در نظر میگیریم و خروجی
	${\hat x}$
	را با نمونه برداری از توزیع
	$a\left( z \right)$
	میسازیم. در این صورت احتمال آن که در خروجی شبکه دیکودر پس از نمونه برداری
	$x$
	 به دست آمده باشد را میتوان با
	 ${P_\theta }\left( {\hat x = x|z} \right)$
	 نمایش داد. حال اگر برای
	 ${\hat x}$
	 یک توزیع برنولی چند متغیره در نظر بگیریم، میتوان این احتمال را به صورت زیر نوشت:
	 $$
	 {P_\theta }\left( {\hat x = x|z} \right) = \prod\limits_{i = 1}^D {{{\left( {{a_i}} \right)}^{{x_i}}}{{\left( {1 - {a_i}} \right)}^{1 - {x_i}}}}
	 $$
	 
	 رابطه فوق چیزی نیست به جز احتمال مشترک چند متغیر برنولی که در آن
	 ${{a_i}}$
	 درایه i ام بردار خروجی
	 $a\left( z \right)$
	 میباشد. با توجه به این توضیحات، میتوان لگاریتم این عبارت را به صورت زیر نوشت:
	 
	 $$
	 \log {P_\theta }\left( {\hat x = x|z} \right) = \log \left[ {\prod\limits_{i = 1}^D {{{\left( {{a_i}} \right)}^{{x_i}}}{{\left( {1 - {a_i}} \right)}^{1 - {x_i}}}} } \right] = \sum\limits_{i = 1}^D {{x_i}\log {a_i} + (1 - {x_i})\log (1 - {a_i})}  =  - BCE\left( {x,a\left( z \right)} \right)
	 $$
	 
	 \item
	 
	 راه حلی که برای این مشکل پیشنهاد میشود، روش Gumble-Softmax است که یک روش تقریبی میباشد. این روش در چند گام انجام میشود که باعث میشود اولا اپراتور احتمالاتی از مسیر اصلی گراف محاسبات کنار رفته و محاسبات به صورت deterministic انجام شود و ثانیا با کنار گذاشتن اپراتورهای گسسته از مسیر اصلی گراف، بتوان backprop  را به درستی انجام داد. برای این منظور فرض کنید یک مسئله نمونه برداری k کلاسه داریم و میخواهیم از توزیع داده شده نمونه بگیریم. همچنین فرض کنید logits هایی که میخواهیم از آن ها نمونه بگیریم را با
	 $\left\{ {{a_i}} \right\}_{i = 1}^k$
	 نشان دهیم. در این صورت این الگوریتم پیشنهاد میکند تا گامهای زیر طی شود:
	 
	 \begin{itemize}
	 	\item
	 	در گام نخست k متغیر
	 	$\left\{ {{k_i}} \right\}_{i = 1}^k$
	 	را به صورت مستقل از توزیع
	 	$uniform(0,1)$
	 	نمونه میگیریم.
	 	
	 	\item
	 	متغیرهای
	 	$\left\{ {{g_i}} \right\}_{i = 1}^k$
	 	را به صورت
	 	${g_i} =  - \log \left( { - \log \left( {{u_i}} \right)} \right)$
	 	تشکیل میدهیم. در این صورت هر یک از متغیرهای
	 	$g_i$
	 	یک توزیع Gumbel	 استاندارد دارند.
	 	
	 	\item
	 	
	 	تا این جا شاخه احتمالاتی تولید متغیر را از گراف اصلی جدا کردیم حالا لازم است تا این متغیرها را با logit هایی که از قبل داشتیم  ترکیب کنیم. برای این منظور جملات
	 	$b_i$
	 	را به صورت
	 	${b_i} = {a_i} + {g_i}$
	 	میسازیم. میتوان نشان داد که متغیر تصادفی
	 	$j = \arg \mathop {\max }\limits_i {b_i}$
	 	دقیقاً همان توزیعی را دارد که ما به دنبال آن بودیم. اما هنوز یک مشکل دیگر باقی مانده و آن استفاده از اپراتور گسسته argmax در وسط شبکه است.
	 	
	 	\item
	 	
	 	برای حل مشکل این اپراتور گسسته، از یک روش تقریبی با کمک لایه Softmax استفاده میکنند. لایه Softmax علاوه بر بردار توزیع احتمالات ورودی، یک پارامتر دیگر نیز به عنوان ورودی دریافت میکند که دما (Temperature) نام دارد. نحوه اعمال این پارامتر روی ورودی به این صورت است که ابتدا همه دادههای ورودی به این پارامتر دما
	 	$\left( \lambda  \right)$
	 	تقسیم میشوند و سپس از Softmax عادی استفاده میشود. پارامتر دما، اثر ویژهای روی شکل بردار خروجی دارد؛ اگر 
	 	$\lambda  = 1$
	 	باشد که همان سافتمکس معمولی خواهد بود اما هرچه
	 	$\lambda$
	 	به صفر نزدیکتر شود، بردار به دست آمده در خروجی به یک بردار one-hot نزدیکتر میشود. در واقع با کوچک شدن این پارامتر به اندازه کافی، تنها درایه با بیشترین مقدار در ورودی برابر یک شده و بقیه به صفر خیلی نزدیک میشوند. همچنین با زیاد شدن
	 	$\lambda$
	 	به سمت بینهایت، مستقل از توزیع ورودی، یک توزیع یکنواخت در خروجی خواهیم داشت. لذا مطلوب ترین حالت همان است که
	 	$\lambda$
	 	تا جای ممکن کوچک انتخاب شود تا تابع argmax به خوبی تقریب زده شود. البته باید توجه کرد که خیلی کوچک گرفتن پارامتر دما میتواند باعث زیاد شدن واریانس گرادیان بازگشتی از این لایه شود.
	 	
	 	
 	\end{itemize}
 
 	\item
 	
 	در
 	$\beta VAE$
 	دقیقاً مشابه
 	$VAE$
 	هدف یادگرفتن فضای نهانی است که بتواند دادها را خوب تولید کند با این تفاوت که در
 	$\beta VAE$
 	تاکید بیشتری روی disentangle بودن فضای نهان صورت میگیرد. به طور دقیقتر،
 	$\beta VAE$
 	نیز همانند
 	$VAE$
 	به دنبال بهینه کردن خطای بازسازی به صورت زیر است:
 	$$
 	\mathop {\max }\limits_{\theta ,\varphi } {\mathbb{E}_{x \sim {P_{Data}}}}\left[ {{\mathbb{E}_{z \sim {q_\varphi }\left( {z|x} \right)}}\left[ {\log {P_\theta }\left( {x|z} \right)} \right]} \right]
 	$$
 	علاوه بر عبارت فوق، لازم است تا یک شرط دیگر به منظور ساده کردن فضای نهان اضافه کنیم. به عبارت دیگر مسئله بهینه سازی
 	$\beta VAE$
 	یک مسئله constrained است که شرط آن روی فضای نهان به صورت زیر خواهد بود:
 	$$
 	\begin{array}{l}
 		\mathop {\max }\limits_{\theta ,\varphi } {\mathbb{E}_{x \sim {P_{Data}}}}\left[ {{\mathbb{E}_{z \sim {q_\varphi }\left( {z|x} \right)}}\left[ {\log {P_\theta }\left( {x|z} \right)} \right]} \right]\\
 		KL\left( {{q_\varphi }\left( {z|x} \right)||P\left( z \right)} \right) < \delta  \Rightarrow KL\left( {{q_\varphi }\left( {z|x} \right)||P\left( z \right)} \right) - \delta  < 0
 	\end{array}
 	$$
 	مسئله فوق یک مسئله بهینه سازی مشروط است که حل کردن آن الزاماً ساده نیست. برای از بین بردن شرط، از تکنیک لاگرانژ و ضریب
 	$\beta$
 	(به عنوان یک هایپرپارمتر) استفاده میشود و قسمت شرط را وارد عبارت بهینه سازی میکنند:
 	$$
 	{\mathbb{E}_{x \sim {P_{Data}}}}\left[ {{\mathbb{E}_{z \sim {q_\varphi }\left( {z|x} \right)}}\left[ {\log {P_\theta }\left( {x|z} \right)} \right]} \right] - \beta \left( {KL\left( {{q_\varphi }\left( {z|x} \right)||P\left( z \right)} \right) - \delta } \right)
 	$$
 	حال توجه کنید که عبارت
 	$\beta \delta$
 	یک مقدار مثبت ثابت و مستقل از پارامترهاست لذا در مجموع میتوان تابع ضرر زیر را برای شبکه نوشت که به عنوان تابع ضرر
 	$\beta VAE$
 	شناخته می‌شود:
 	$$
 	L\left( {\theta ,\varphi } \right) =  - {\mathbb{E}_{x \sim {P_{Data}}}}\left[ {{\mathbb{E}_{z \sim {q_\varphi }\left( {z|x} \right)}}\left[ {\log {P_\theta }\left( {x|z} \right)} \right]} \right] + \beta KL\left( {{q_\varphi }\left( {z|x} \right)||P\left( z \right)} \right)
 	$$
\end{enumerate}
%}




%\section*{سوالات عملی (5 + 40 نمره)}
%\مسئله{(20 نمره)}
%هدف این سوال طراحی یک شبکه ساده GAN می‌باشد که بتواند تصاویر دادگان MNIST را تولید نماید. فایل GAN.ipynb را براساس موارد خواسته شده تکمیل نمایید. دقت کنید که بخش هایی از نمره بستگی به نتایج بدست آمده دارند. باتوجه به نکات تمرین و مواردی که در کلاس عنوان شده،‌ ساختار شبکه و تابع خطا و پارامترهای دیگر را طوری طراحی و انتخاب نمایید که در نهایت تصاویر باکیفیتی توسط Generator تولید شود و فرایند آموزش پایدار باشد.
در نهایت فایل تکمیل شده به همراه نتایج را بعلاوه فایل پارامترهای شبکه Generator ای که آموزش داده‌اید به همراه دیگر بخش‌های تمرین ارسال نمایید.
%
%
%\مسئله{(5 + 20 نمره)}
%
VAE Conditional یکی از ورژن‌های modified شده VAE بوده که بر خلاف VAE کلاسیک، متغیرهای مورد نیاز را به صورت conditioned نسبت به برخی متغیرهای تصادفی تخمین می‌زند. در این تمرین هدف مقایسه خروجی این دو مدل بر روی مجموعه داده MNIST می‌باشد. لطفا کد هر دو روش پیاده‌سازی شده و خروجی آن‌ها از بعد میزان وضوح تصاویر تولید شده مقایسه گردد.

برای طراحی شبکه های VAE و CVAE ، به طور کلی محدودیت چندانی وجود ندارد اما پیشنهاد می شود برای قسمت Encoder ، سه لایه کانولوشن دو بعدی به ترتیب با 16، 32، و 32 لایه به همراه MaxPool دو بعدی 2 در 2 پس از هرکدام طراحی شود. برای قسمت Decoder نیز دو لایه خطی به ترتیب با 32 و 64 لایه طراحی گردد. برای تایع هزینه لطفا از Entropy Cross Binary به همراه Divergence KL استفاده شود. برای Optimizer نیز از Adam استفاده گردد. مابقی پارامترها همانند mean می‌تواند به صورت customize شده انتخاب گردد و بسته به خروجی بهتر تغییر کند. برای راهنمایی بیشتر می‌توانید از کد موجود در
\href{https://github.com/chendaichao/VAE-pytorch}{این لینک}
استفاده کنید. لطفا کد را کپی نکرده و صرفا برای کمک و الهام‌گیری کد زنی خود از آن استفاده شود. توجه شود حتی ساختار پیشنهادی شبکه بسته به صلاح دید شخصی شما قابل تغییر بوده فقط توجه شود که نوشتن گزارش بخش عملی الزامی بوده و دارای نمره می‌باشد لذا حتما تمامی مراحل اعم از ساختار شبکه‌ها و پارامترها باید به طور کامل در گزارش توضیح داده شوند. برای پیاده‌سازی نیز تنها مجاز به استفاده از کتابخانه pytorch می‌باشید.
\\
\\


توضیحات کلی:


به وضوح مشورت و همفکری در حل سوالات هیچگونه ای مشکلی ایجاد نخواهد کرد اما جواب سوالات به هیچ عنوان نباید یکسان باشد. هر فرد باید جداگانه و Unique پاسخ‌های خود را بنویسد و در صورت شباهت بسیار زیاد نمره بین افرادی که پاسخ های بسیار مشابه دارند تقسیم خواهد گشت.
در صورتی که کد شما در بخش عملی به خطا برخورد و خطا قابل برطرف کردن نبود، نمره شما بر اساس کیفیت کد به میزان قابل قبولی لحاظ خواهد گشت. لذا لطفا پاسخ سوال‌های عملی را خالی نگذارید.
برای سوال عملی لطفا کد و گزارش هردو ارسال شوند. کمبود یکی از آن‌ها منجر به از دست رفتن نمره خواهد شد.


%


\begin{flushleft}
	موفق باشید :)
\end{flushleft}



\پایان{نوشتار}
